% Options for packages loaded elsewhere
% Options for packages loaded elsewhere
\PassOptionsToPackage{unicode}{hyperref}
\PassOptionsToPackage{hyphens}{url}
\PassOptionsToPackage{dvipsnames,svgnames,x11names}{xcolor}
%
\documentclass[
  letterpaper,
  DIV=11,
  numbers=noendperiod]{scrartcl}
\usepackage{xcolor}
\usepackage{amsmath,amssymb}
\setcounter{secnumdepth}{-\maxdimen} % remove section numbering
\usepackage{iftex}
\ifPDFTeX
  \usepackage[T1]{fontenc}
  \usepackage[utf8]{inputenc}
  \usepackage{textcomp} % provide euro and other symbols
\else % if luatex or xetex
  \usepackage{unicode-math} % this also loads fontspec
  \defaultfontfeatures{Scale=MatchLowercase}
  \defaultfontfeatures[\rmfamily]{Ligatures=TeX,Scale=1}
\fi
\usepackage{lmodern}
\ifPDFTeX\else
  % xetex/luatex font selection
\fi
% Use upquote if available, for straight quotes in verbatim environments
\IfFileExists{upquote.sty}{\usepackage{upquote}}{}
\IfFileExists{microtype.sty}{% use microtype if available
  \usepackage[]{microtype}
  \UseMicrotypeSet[protrusion]{basicmath} % disable protrusion for tt fonts
}{}
\makeatletter
\@ifundefined{KOMAClassName}{% if non-KOMA class
  \IfFileExists{parskip.sty}{%
    \usepackage{parskip}
  }{% else
    \setlength{\parindent}{0pt}
    \setlength{\parskip}{6pt plus 2pt minus 1pt}}
}{% if KOMA class
  \KOMAoptions{parskip=half}}
\makeatother
% Make \paragraph and \subparagraph free-standing
\makeatletter
\ifx\paragraph\undefined\else
  \let\oldparagraph\paragraph
  \renewcommand{\paragraph}{
    \@ifstar
      \xxxParagraphStar
      \xxxParagraphNoStar
  }
  \newcommand{\xxxParagraphStar}[1]{\oldparagraph*{#1}\mbox{}}
  \newcommand{\xxxParagraphNoStar}[1]{\oldparagraph{#1}\mbox{}}
\fi
\ifx\subparagraph\undefined\else
  \let\oldsubparagraph\subparagraph
  \renewcommand{\subparagraph}{
    \@ifstar
      \xxxSubParagraphStar
      \xxxSubParagraphNoStar
  }
  \newcommand{\xxxSubParagraphStar}[1]{\oldsubparagraph*{#1}\mbox{}}
  \newcommand{\xxxSubParagraphNoStar}[1]{\oldsubparagraph{#1}\mbox{}}
\fi
\makeatother


\usepackage{longtable,booktabs,array}
\usepackage{calc} % for calculating minipage widths
% Correct order of tables after \paragraph or \subparagraph
\usepackage{etoolbox}
\makeatletter
\patchcmd\longtable{\par}{\if@noskipsec\mbox{}\fi\par}{}{}
\makeatother
% Allow footnotes in longtable head/foot
\IfFileExists{footnotehyper.sty}{\usepackage{footnotehyper}}{\usepackage{footnote}}
\makesavenoteenv{longtable}
\usepackage{graphicx}
\makeatletter
\newsavebox\pandoc@box
\newcommand*\pandocbounded[1]{% scales image to fit in text height/width
  \sbox\pandoc@box{#1}%
  \Gscale@div\@tempa{\textheight}{\dimexpr\ht\pandoc@box+\dp\pandoc@box\relax}%
  \Gscale@div\@tempb{\linewidth}{\wd\pandoc@box}%
  \ifdim\@tempb\p@<\@tempa\p@\let\@tempa\@tempb\fi% select the smaller of both
  \ifdim\@tempa\p@<\p@\scalebox{\@tempa}{\usebox\pandoc@box}%
  \else\usebox{\pandoc@box}%
  \fi%
}
% Set default figure placement to htbp
\def\fps@figure{htbp}
\makeatother





\setlength{\emergencystretch}{3em} % prevent overfull lines

\providecommand{\tightlist}{%
  \setlength{\itemsep}{0pt}\setlength{\parskip}{0pt}}



 


\KOMAoption{captions}{tableheading}
\makeatletter
\@ifpackageloaded{caption}{}{\usepackage{caption}}
\AtBeginDocument{%
\ifdefined\contentsname
  \renewcommand*\contentsname{Table of contents}
\else
  \newcommand\contentsname{Table of contents}
\fi
\ifdefined\listfigurename
  \renewcommand*\listfigurename{List of Figures}
\else
  \newcommand\listfigurename{List of Figures}
\fi
\ifdefined\listtablename
  \renewcommand*\listtablename{List of Tables}
\else
  \newcommand\listtablename{List of Tables}
\fi
\ifdefined\figurename
  \renewcommand*\figurename{Figure}
\else
  \newcommand\figurename{Figure}
\fi
\ifdefined\tablename
  \renewcommand*\tablename{Table}
\else
  \newcommand\tablename{Table}
\fi
}
\@ifpackageloaded{float}{}{\usepackage{float}}
\floatstyle{ruled}
\@ifundefined{c@chapter}{\newfloat{codelisting}{h}{lop}}{\newfloat{codelisting}{h}{lop}[chapter]}
\floatname{codelisting}{Listing}
\newcommand*\listoflistings{\listof{codelisting}{List of Listings}}
\makeatother
\makeatletter
\makeatother
\makeatletter
\@ifpackageloaded{caption}{}{\usepackage{caption}}
\@ifpackageloaded{subcaption}{}{\usepackage{subcaption}}
\makeatother
\usepackage{bookmark}
\IfFileExists{xurl.sty}{\usepackage{xurl}}{} % add URL line breaks if available
\urlstyle{same}
\hypersetup{
  pdftitle={Curriculum Vitae},
  colorlinks=true,
  linkcolor={blue},
  filecolor={Maroon},
  citecolor={Blue},
  urlcolor={Blue},
  pdfcreator={LaTeX via pandoc}}


\title{Curriculum Vitae}
\author{}
\date{}
\begin{document}
\maketitle


\href{assets/pdf/frimpong-joseph-cv-resume.pdf}{\textbf{Download PDF
Version}}

\subsection{Contact Information}\label{contact-information}

\textbf{Email:} frimpsjoe@wayne.edu\\
\textbf{Institution:} NST, Argonne National Laboratory\\
\textbf{GitHub:} \href{https://github.com/frimpsjoek}{frimpsjoek}\\
\textbf{LinkedIn:} \href{https://linkedin.com/in/frimpsjoek}{frimpsjoek}

\subsection{Current Position}\label{current-position}

\textbf{Postdoctoral Researcher} \textbar{} \emph{Present}\\
NST Division, Argonne National Laboratory

\begin{itemize}
\tightlist
\item
  Research on electronic structure and optical properties of materials
\item
  Applications of DFT, GW/BSE, and machine learning to catalysis and
  energy materials\\
\item
  Development of computational workflows for materials discovery
\end{itemize}

\subsection{Research Interests}\label{research-interests}

\subsubsection{Computational Methods}\label{computational-methods}

\begin{itemize}
\tightlist
\item
  Density Functional Theory (DFT), many-body perturbation theory
  (GW/BSE)
\item
  Electronic structure and optical properties calculations
\item
  Applications in catalysis, energy, and optoelectronic materials
\end{itemize}

\subsubsection{AI \& Machine Learning}\label{ai-machine-learning}

\begin{itemize}
\tightlist
\item
  High-throughput materials discovery
\item
  Enhanced electronic structure methods
\item
  Structure-property relationship predictions
\end{itemize}

\subsubsection{Materials Science}\label{materials-science}

\begin{itemize}
\tightlist
\item
  Heterogeneous interfaces and quantum dots
\item
  Nanoscale materials for energy-efficient devices
\end{itemize}

\subsection{Technical Skills}\label{technical-skills}

\textbf{Programming Languages}\\
Python, Bash/Shell Scripting, HTML/CSS, Git version control

\textbf{Computational Methods}\\
DFT, GW/BSE methods, Electronic structure calculations, High-performance
computing

\textbf{Systems \& Tools}\\
Linux, Git/GitHub, High-performance computing environments

\subsection{Publications}\label{publications}

Full publication list available on \href{publications.qmd}{Publications
page}.

\subsection{Research Projects}\label{research-projects}

Current and past research projects detailed on
\href{projects.qmd}{Projects page}.

\subsection{Professional Activities}\label{professional-activities}

\textbf{Academic Writing}\\
Maintaining research blog ``Reflections of a young academic'' sharing
insights on academic life and computational materials science.

\textbf{Open Source Contributions}\\
Contributing to computational chemistry and materials science software
packages on GitHub.




\end{document}
